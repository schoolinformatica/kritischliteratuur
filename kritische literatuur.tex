\documentclass{article}

\title{Is een Artificial Intelligence ethisch verantwoord?}
\date{19-1-2017}
\author{Nolen, Cees-Jan\\
  \texttt{0902130}
  \and
  Schenk, Steven\\
  \texttt{0894490}}


\usepackage{fancyhdr}
\usepackage{parskip}
\usepackage{cite}
\RequirePackage{url}



\pagestyle{fancy}
\fancyhead[L]{Is een AI ethisch verantwoord?}


\begin{document}
\pagenumbering{gobble}
\maketitle
\newpage
\pagenumbering{arabic}
\tableofcontents

\newpage
\section{Doel en doelgroep}
Voor onze stage bij de Universiteit Tilburg maken wij een plugin voor Unity\footnote{Unity is een game engine voor het maken van crossplatform 2D en 3D games.}. 
Deze plugin moet het mogelijk maken om
makkelijk Non Playable Character's (NPC) te genereren, met ieder een eigen persoonlijkheid. Echter moet de NPC het adaptatievermogen hebben om zich
op runtime\footnote{Tijdens het draaien van de software/game}  aan situaties aan te kunnen passen. Dit betekent dat de NPC zelf na moet denken en beslissingen moet 
kunnen maken op basis van zijn persoonlijkheidsprofiel. Een NPC wordt aangestuurd door de computer. De computer moet dus zelf kunnen "denken"
over het maken van beslissingen. Een computer die autonoom kan denken, wordt ook wel een kunstmatige intelligentie genoemd, of in het Engels,
een Artificial Intelligence (AI).

Echter is een AI een mondiaal besproken onderwerp, en zijn de meningen er over zeer controversieel. Menig mens is er van overtuigd dat het
zoveelste Hollywood doom scenario, waarin de robots en computers de wereld overnemen, werkelijkheid kan worden. Anderen, staan hier weer
lijnrecht tegenover en beweren dat een AI de wereld een betere plaats kan maken.  

Voor dit artikel is geen voorkennis vereist betreft de werking van dergelijke AI's. Definities en vakjargon zullen in dermate worden uitgelegd
dat het artikel voor iedere lezer toegankelijk moet zijn. 


\newpage
\section{Thema en inhoud}
Het thema in dit artikel luidt: "Is een Artificial Intelligence ethisch verantwoord?". Deze vraag zal centraal staan in het artikel. 
De hoofdvraag is hier dus ook een weerspiegeling van. De voor- en nadelen en meningen van een AI zullen in dit arikel belicht worden
om tot een nauwkeurig maar subjectief antwoord te komen. 

Door het beantwoorden van deelvragen willen wij structureel tot een antwoord komen. De deelvragen zullen een gedetailleerd beeld moeten
geven op het antwoord van de hoofdvraag. Hiervoor hebben wij de volgende deelvragen
gedefinieerd:


\begin{enumerate}
	\item Waarom is de AI onze vriend?
	\item Waarom is de AI onze vijand?
\end{enumerate}  

\newpage
\section{Abstract}
	Artficial Intelligence (AI) is een mondiaal besproken onderwerp waar iedereen zijn eigen mening over heeft. 
	Maken ze ons leven makkelijker, door processen te automatiseren en nauwkeuriger te werk te gaan, of
	maken ze ons juist overbodig door al onze taken over te nemen? Hebben ze in Hollywood dan toch gelijk
	en leven we binnen 100 jaar in een tijdperk waar de robots de macht hebben? In dit artikel willen wij
	een stap zetten in het beantwoorden van deze vragen.

\section{Waarom is de AI onze vriend?}
De sci-fi films van vroeger waarin alles bestuurt wordt door robots en de computer je beste vriend kan zijn
lijkt helemeel niet meer zo sci-fi als het was. Deze films beginnen langzamerhand de werkelijkheid te worden. 
Computers worden niet alleen steeds slimmer, maar beginnen ook een steeds humanere vorm aan te nemen. Waar we 
al tegen onze telefoons kunnen praten zal het niet lang meer duren voor je in een zelfrijdende auto naar je werk rijdt. 

Google gaat nog een stap verder en maakt misschien ons brein wel overbodig. Door het toepassen van de nieuwe techniek genaamd
"deep learning"\footnote{Computers zijn zelflerend door het toepassen van algoritmes die objecten van hoog abstractieniveau
om kunnen zetten in data} is het mogelijk om de biologische structuur van onze hersenen te reproduceren in software.
Hierdoor wordt de onafhankelijkheid van de computer nog een stap groter. Met deep learning zijn computers
zelflerend en zijn hierdoor niet meer afhankelijk van de mens voor het vergaren van kennis in de vorm van data. Google zal
dus steeds slimmer worden, waardoor de mens steeds minder hoeft te weten. Immers, we kunnen Google al bijna als ons
tweede brein beschouwen.\cite{breinoverbodig}

Ook voor medische doeleinden is de computer niet meer weg te denken. Een goed voorbeeld hiervan is Stephen Hawking, die ondanks
zijn verlamming, alsnog kan praten. Door het gebruik van een computer die Stephen met zijn kaak kan bedienen kan hij een
tekst samenstellen die vervolgens uitgesproken wordt door de bekende robot achtige stem. Uiteraard is het niet zomaar mogelijk
om voor hem snel een tekst te typen. Door het gebruik van slimme algoritmes kan de computer voorspellen wat Stephen wilt gaan
zeggen. Stephen hoeft hier door maar enkele letters van een woord te typen waarna de computer het voor hem aanvult.\cite{stephenhawking}

Niet alleen in de nazorg, maar ook tijdens medische procedurs of beslissingen kan een AI uitkomst bieden. Waar in menig ziekenhuis
nog wel eens medische fouten worden gemaakt, als gevolg van een menselijke blunder, maakt een computer, mits er geen fouten in de software
aanwezig zijn, geen fouten.\cite{computerziekenhuis} Een computer heeft een vaste instructieset en kan niet van deze routine afwijken, waar
een mens nog wel eens een steekje laat vallen. Ook bij chirurgische ingrepen speelt een computer parte.\cite{medischerobots} Chirurgen kunnen
robots gebruiken die fungeren als hun armen tijdens medische ingrepen. Hierdoor kunnen ze veel nauwkeuriger te werk gaan. Daarnaast zijn er
ook al robots die autonoom te werk gaan. Een voorbeeld hiervan is de veebot. Deze kan bloed afnemen bij patiënten, zonder een ader te missen.\cite{veebot}  







\section{Waarom is de AI onze vijand?}
Er zijn verschillende redenen te noemen waarom we AI niet met open armen zouden moeten ontvangen. Deze redenen komen in deze deelvraag aan bod.
Een reden die al snel naar voren komt, is dat de angst op werkloosheid zal toenemen. Dit is al eerder gebeurd in de geschiedenis, in de Industriële Revolutie, toen de machine werd uitgevonden en vele fabrieksarbeiders op straat kwamen staan, de mensen vrezen met AI het zelfde efffect. BRONNEN Minister Ascher evenals een onderzoek door van instituut Rathenau tonen aan dat er een degelijke angst is ontstaan naar mate er meer door AI gedaan wordt. 
Een andere reden die vaak wordt genoemd is dat er een zekere angst ontstaat voor robots. Naarmate er meer gerobotiseerd zal worden, zal eveneens de angst groeien dat robots meer en meer op de wereld zullen gaan leven en in zekere mate de wereld zullen overnemen. 
Nog een reden die vaak aangedragen wordt is de angst op vervreemding. Doordat er steeds meer sprake zal zijn van AI, verwachten de mensen dat ze vervreemd raken met een AI. 

\section{Conclusie}
Uit de deelvragen is het volgende naar voren gekomen; 

\newpage
\nocite{robotsamenleving,vriendofvijand,arbeidsmarkt,breinoverbodig,stephenhawking,killemachine,uitgeroeid,autonomous,benificialai,pastandfuture,computerziekenhuis}
\bibliographystyle{IEEEtran}
\bibliography{IEEEabrv,bib}
% \bibliography{literatuur}
% \bibliographystyle{siam}

\end{document}