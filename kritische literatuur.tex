\documentclass{article}

\title{Is een Artificial Intelligence ethisch verantwoord?}
\date{19-9-2016}
\author{Nolen, Cees-Jan\\
  \texttt{0902130}
  \and
  Schenk, Steven\\
  \texttt{0894490}}


\usepackage{fancyhdr}
\usepackage{parskip}
\usepackage{cite}
\RequirePackage{url}



\pagestyle{fancy}
\fancyhead[L]{Is een AI ethisch verantwoord?}


\begin{document}
\pagenumbering{gobble}
\maketitle
\newpage
\pagenumbering{arabic}
\tableofcontents

\newpage
\section{Doel en doelgroep}
Voor onze stage bij de Universiteit Tilburg maken wij een plugin voor Unity\footnote{Unity is een game engine voor het maken van crossplatform 2D en 3D games.}. 
Deze plugin moet het mogelijk maken om
makkelijk Non Playable Character's (NPC) te genereren, met ieder een eigen persoonlijkheid. Echter moet de NPC het adaptatievermogen hebben om zich
op runtime\footnote{Tijdens het draaien van de software/game}  aan situaties aan te kunnen passen. Dit betekent dat de NPC zelf na moet denken en beslissingen moet 
kunnen maken op basis van zijn persoonlijkheidsprofiel. Een NPC wordt aangestuurd door de computer. De computer moet dus zelf kunnen "denken"
over het maken van beslissingen. Een computer die autonoom kan denken, wordt ook wel een kunstmatige intelligentie genoemd, of in het Engels,
een Artificial Intelligence (AI).

Echter is een AI een mondiaal besproken onderwerp, en zijn de meningen er over zeer controversieel. Menig mens is er van overtuigd dat het
zoveelste Hollywood doom scenario, waarin de robots en computers de wereld overnemen, werkelijkheid kan worden. Anderen, staan hier weer
lijnrecht tegenover en beweren dat een AI de wereld een betere plaats kan maken.  

Voor dit artikel is geen voorkennis vereist betreft de werking van dergelijke AI's. Definities en vakjargon zullen in dermate worden uitgelegd
dat het artikel voor iedere lezer toegankelijk moet zijn. 


\newpage
\section{Thema en inhoud}
Het thema in dit artikel luidt: "Is een Artificial Intelligence ethisch verantwoord?". Deze vraag zal centraal staan in het artikel. 
De hoofdvraag is hier dus ook een weerspiegeling van. De voor- en nadelen en meningen van een AI zullen in dit arikel belicht worden
om tot een nauwkeurig maar subjectief antwoord te komen. 

Door het beantwoorden van deelvragen willen wij structureel tot een antwoord komen. De deelvragen zullen een gedetailleerd beeld moeten
geven op het antwoord van de hoofdvraag. Hiervoor hebben wij de volgende deelvragen
gedefinieerd:


\begin{enumerate}
	\item Waarom is de AI onze vriend?
	\item Waarom is de AI onze vijand?
\end{enumerate}  

\newpage
\section{Abstract}
	Artficial Intelligence (AI) is een mondiaal besproken onderwerp waar iedereen zijn eigen mening over heeft. 
	Maken ze ons leven makkelijker, door processen te automatiseren en nauwkeuriger te werk te gaan, of
	maken ze ons juist overbodig door al onze taken over te nemen? Hebben ze in Hollywood dan toch gelijk
	en leven we binnen 100 jaar in een tijdperk waar de robots de macht hebben? In dit artikel willen wij
	een stap zetten in het beantwoorden van deze vragen.

\section{Waarom is de AI onze vriend?}


\section{Waarom is de AI onze vijand?}

\section{Conclusie}

\newpage
\nocite{robotsamenleving,vriendofvijand,arbeidsmarkt,breinoverbodig,stephenhawking,killemachine,uitgeroeid,autonomous,benificialai,pastandfuture}
\bibliographystyle{IEEEtran}
\bibliography{IEEEabrv,bib}
% \bibliography{literatuur}
% \bibliographystyle{siam}

\end{document}