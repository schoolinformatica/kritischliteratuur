\documentclass{article}

\title{Is een Artificial Intelligence ethisch verantwoord?}
\date{19-1-2017}
\author{Nolen, Cees-Jan\\
  \texttt{0902130}
  \and
  Schenk, Steven\\
  \texttt{0894490}}


\usepackage{fancyhdr}
\usepackage{parskip}
\usepackage{cite}
\RequirePackage{url}



\pagestyle{fancy}
\fancyhead[L]{Is een AI ethisch verantwoord?}


\begin{document}
\pagenumbering{gobble}
\maketitle
\newpage
\pagenumbering{arabic}
\tableofcontents

\newpage
\section{Doel en doelgroep}
Voor onze stage bij de Universiteit Tilburg maken wij een plugin voor Unity\footnote{Unity is een game engine voor het maken van crossplatform 2D en 3D games.}. 
Deze plugin moet het mogelijk maken om
makkelijk Non Playable Character's (NPC) te genereren, met ieder een eigen persoonlijkheid. Echter moet de NPC het adaptatievermogen hebben om zich
op runtime\footnote{Tijdens het draaien van de software/game}  aan situaties aan te kunnen passen. Dit betekent dat de NPC zelf na moet denken en beslissingen moet 
kunnen maken op basis van zijn persoonlijkheidsprofiel. Een NPC wordt aangestuurd door de computer. De computer moet dus zelf kunnen "denken"
over het maken van beslissingen. Een computer die autonoom kan denken, wordt ook wel een kunstmatige intelligentie genoemd, of in het Engels,
een Artificial Intelligence (AI).

Echter is een AI een mondiaal besproken onderwerp, en zijn de meningen er over zeer controversieel. Menig mens is er van overtuigd dat het
zoveelste Hollywood doom scenario, waarin de robots en computers de wereld overnemen, werkelijkheid kan worden. Anderen, staan hier weer
lijnrecht tegenover en beweren dat een AI de wereld een betere plaats kan maken.  

Voor dit artikel is geen voorkennis vereist betreft de werking van dergelijke AI's. Definities en vakjargon zullen in dermate worden uitgelegd
dat het artikel voor iedere lezer toegankelijk moet zijn. 


\newpage
\section{Thema en inhoud}
Het thema in dit artikel luidt: "Is een Artificial Intelligence ethisch verantwoord?". Deze vraag zal centraal staan in het artikel. 
De hoofdvraag is hier dus ook een weerspiegeling van. De voor- en nadelen en meningen van een AI zullen in dit arikel belicht worden
om tot een nauwkeurig maar subjectief antwoord te komen. 

Door het beantwoorden van deelvragen willen wij structureel tot een antwoord komen. De deelvragen zullen een gedetailleerd beeld moeten
geven op het antwoord van de hoofdvraag. Hiervoor hebben wij de volgende deelvragen
gedefinieerd:


\begin{enumerate}
	\item Waarom is de AI onze vriend?
	\item Waarom is de AI onze vijand?
\end{enumerate}  

\newpage
\section{Abstract}
	Artficial Intelligence (AI) is een mondiaal besproken onderwerp waar iedereen zijn eigen mening over heeft. 
	Maken ze ons leven makkelijker, door processen te automatiseren en nauwkeuriger te werk te gaan, of
	maken ze ons juist overbodig door al onze taken over te nemen? Hebben ze in Hollywood dan toch gelijk
	en leven we binnen 100 jaar in een tijdperk waar de robots de macht hebben? In dit artikel willen wij
	een stap zetten in het beantwoorden van deze vragen.

\section{Waarom is de AI onze vriend?}
De sci-fi films van vroeger waarin alles bestuurt wordt door robots en de computer je beste vriend kan zijn
lijkt helemeel niet meer zo sci-fi als het was. Deze films beginnen langzamerhand de werkelijkheid te worden. 
Computers worden niet alleen steeds slimmer, maar beginnen ook een steeds humanere vorm aan te nemen. Waar we 
al tegen onze telefoons kunnen praten zal het niet lang meer duren voor je in een zelfrijdende auto naar je werk rijdt. 

Google gaat nog een stap verder en maakt misschien ons brein wel overbodig. Door het toepassen van de nieuwe techniek genaamd
"deep learning"\footnote{Computers zijn zelflerend door het toepassen van algoritmes die objecten van hoog abstractieniveau
om kunnen zetten in data} is het mogelijk om de biologische structuur van onze hersenen te reproduceren in software.
Hierdoor wordt de onafhankelijkheid van de computer nog een stap groter. Met deep learning zijn computers
zelflerend en zijn hierdoor niet meer afhankelijk van de mens voor het vergaren van kennis in de vorm van data. Google zal
dus steeds slimmer worden, waardoor de mens steeds minder hoeft te weten. Immers, we kunnen Google al bijna als ons
tweede brein beschouwen.\cite{breinoverbodig}







\section{Waarom is de AI onze vijand?}
Er zijn verschillende redenen te noemen waarom we AI niet met open armen zouden moeten ontvangen. Deze redenen komen in deze deelvraag aan bod.
Een reden die al snel naar voren komt, is dat de angst op werkloosheid zal toenemen. Dit is al eerder gebeurd in de geschiedenis, in de Industriële Revolutie, toen de machine werd uitgevonden en vele fabrieksarbeiders op straat kwamen staan, de mensen vrezen met AI het zelfde efffect. Minister Ascher\cite{vriendofvijand} evenals een onderzoek door van instituut Rathenau\cite{vriendofvijand}  tonen aan dat er een degelijke angst is ontstaan naar mate er meer door AI gedaan wordt. Ook verschillende adviesbureau's tonen aan dat er door innovaties van robots en andere AI maar liefst twee tot drie miljoen banen in gevaar komen. Dit betreft voornamelijk mensen in de laagopgeleide sector m.b.t. dienstverlening en productiegebied. Weer uit een ander onderzoek\cite{vriendofvijand} blijkt dat bijna de helft van alle banen in de komende 30 jaar geautomatiseerd zal worden. Onderzoeken als deze zorgen voor angst onder de mensen. Door deze angsten in te perken worden er verschillende regels gekoppeld aan het gebruik van AI. Toch weerhoudt dit bedrijven er niet van om geen AI in te zetten binnen hun bedrijf. Zo worden in steeds meer supermarkten de kassamedewerkers vervangen door zelfscanners, telefonistes door spraak herkenningssoftware en beveiligers door slimme alarmsystemen. Over enkele jaren als de zelfrijdende auto definitief wordt, is het de vraag wat er gaat gebeuren met de vrachtwagen sector. Steeds meer worden door AI gedigitaliseerd, wat een angst oplevert bij de mensen. Bij de angst op werkloosheid valt als laatste op te merken dat het werk waarvoor de arbeiders lager opgeleid zijn, eerder geautomatiseerd zal worden dan het werk waarbij een hogere opleiding voor vereist is. 

Een andere reden die vaak wordt genoemd is dat er een zekere angst ontstaat voor robots. Naarmate er meer gerobotiseerd zal worden, zal eveneens de angst groeien dat robots meer en meer op de wereld zullen gaan leven en in zekere mate de wereld zullen overnemen. Wij als mensen hebben alles graag onder controle en geven dit niet graag uit handen. Dit is niet alleen in kleine kringen aanwezig maar ook op groter gebied. Zo wordt er bij defensie steeds meer geautomatiseerd waardoor wapens automatisch doelen zoeken en raketten afschieten. Wat op te merken is dat we deze systemen graag gebruiken, maar deze te snel in zetten zonder ze goed te kennen, op deze manier kunnen er grote fouten onstaan. Te denken valt aan de bekende beursfalen\cite{vriendofvijand} waarbij door verkeerde werking van een computer de beurs enorm kelderde. Door zulke falen vertrouwen we niet graag op deze computers. Daarnaast is een computer meer gericht op techniek dan dat het rekening zou houden met mensen. Als de computer voor grote berekeningen meer rekenkracht nodig heeft en dit enkel kan bereiken door het uitroeien van mensheid, dan zal een computer dit zomaar doen, als het hiertoe in staat is. Wij als natuurlijke mensen zullen bij zulke berekeningen deze optie zelfs niet in gedachten nemen. Toch zal AI een rol krijgen in de samenleving, hier kunnen we niet omheen. Hierbij zullen we rekening moeten houden met het feit dat we eigenlijk nooit eerder de wereld hebben kunnen controleren. Het is een typisch rationeel denkbeeld. Dit beeld zien we ook eerder terug in de geschiedenis, namelijk met de Verlichting. Ook toen werd gedacht dat alles aan wetmatigheden is gebonden.

Nog een reden die vaak aangedragen wordt is de angst op vervreemding. Doordat er steeds meer sprake zal zijn van AI, zullen mensen vervreemd raken met natuur en de echte producten die gemaakt worden. Om vervreemding op de juiste manier op te vatten, wordt dit begrip eerst uitgelegd. Vervreemding wordt door Karl Marx\cite{vriendofvijand}, een filosoof, uitgelegd als een proces waarbij de mens zich vervreemd van iets, en tegelijk iets van de mens ontvreemd wordt. Bij een AI zal dit steeds meer voorkomen dat de uitvoerder steeds meer vervreemd wordt van zijn eerdere, nog niet geautomatiseerde, werk. Een voorbeeld hiervan is bijvoorbeeld een boer. Door dat er steeds meer producten komen waarmee de boer zijn werk kan doen, denk aan melk robots of drones die zijn land besproeien, wordt de boer steeds verder ontvreemd van zijn oudere werk. Een ander voorbeeld is dat de mens steeds minder leert communiceren met mensen. In plaats hiervan communiceert men met robots en computers. Whatsapp is hierin een tussenkomst tussen de mensheid en de virtuele wereld. 

\section{Conclusie}
Uit de deelvragen is het volgende naar voren gekomen; 

\newpage
\nocite{robotsamenleving,vriendofvijand,arbeidsmarkt,breinoverbodig,stephenhawking,killemachine,uitgeroeid,autonomous,benificialai,pastandfuture}
\bibliographystyle{IEEEtran}
\bibliography{IEEEabrv,bib}
% \bibliography{literatuur}
% \bibliographystyle{siam}

\end{document}